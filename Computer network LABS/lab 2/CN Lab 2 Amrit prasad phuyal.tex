\documentclass[a4paper,12pt]{article}
\usepackage{geometry}
 \geometry{
 a4paper,
 total={170mm,257mm},
 left=20mm,
 top=20mm,
 }
 \usepackage[export]{adjustbox}
\usepackage[english]{babel}
\usepackage[utf8]{inputenc}
\usepackage{fancyhdr}
\usepackage{multicol}
\pagestyle{fancy}
\fancyhf{}
\rhead{\textit{LAB -2}}
\lhead{\textit{Pul074BEX004}}
\rfoot{\thepage}


\usepackage{mathpazo} % Palatino font
\usepackage{graphicx}
\usepackage{float}

%%%% Anser environment use %%%% Anser environment use %%%% Anser environment use \input{./AnsENV.tex}
%% use \begin{A... {**** argument***}
\RequirePackage{scrextend}

\newenvironment{A}[1]{\textit{Answer:}{\begin{addmargin}[2em]{2em}{#1}\end{addmargin} 
  }}

% just leave some space   
%% use \begin{A... {**** argument***}
\RequirePackage{scrextend}

\newenvironment{A}[1]{\textit{Answer:}{\begin{addmargin}[2em]{2em}{#1}\end{addmargin} 
  }}

% just leave some space   
%% use \begin{A... {**** argument***}
\RequirePackage{scrextend}

\newenvironment{A}[1]{\textit{Answer:}{\begin{addmargin}[2em]{2em}{#1}\end{addmargin} 
  }}

% just leave some space    %% Answer environment 

%%% Question Environment%%%  use 
%%% Question Environment%%%  use 
%%% Question Environment%%%  use \input{./QueENV.tex}   to include
%% Use \begin{Q}....\end{Q}

\newcounter{QC}
\setcounter{QC}{1}
\newenvironment{Q}[1]{
    \section{Question -\arabic{QC}} \stepcounter{QC}{\large\textbf{#1}}
}

%%% Question Environment%%%

   to include
%% Use \begin{Q}....\end{Q}

\newcounter{QC}
\setcounter{QC}{1}
\newenvironment{Q}[1]{
    \section{Question -\arabic{QC}} \stepcounter{QC}{\large\textbf{#1}}
}

%%% Question Environment%%%

   to include
%% Use \begin{Q}....\end{Q}

\newcounter{QC}
\setcounter{QC}{1}
\newenvironment{Q}[1]{
    \section{Question -\arabic{QC}} \stepcounter{QC}{\large\textbf{#1}}
}

%%% Question Environment%%%

 %% Question Environment 
%%%%%% include  Titles.%%%% use \input{./CP}%%%
%%%use """"""""    \CP{}{}{}{}   """" %%%% and 4 argument to craete Title page 
%%%%%%%%%%%%%%%%%%%%%%%%%%%%%%%%%%%%%%%%%%%%%%%%%%%%%%%%%%%%%%%%%
%%%argument number
%% 1=major header ## Course name 
%% 2=minor4 heading ## lab/assignmet no
%% 3=Title  ## Assignment or Lab title
%% 4=submitted to::## input receiver Name"
%%%%%%%%%%%%%%%%%%%%%%%%%%%%%%%%%%%%%%%%%%%%%%%%%%%%%%%%%%%%%%%%%


\usepackage{mathpazo} % Palatino font
\usepackage{graphicx}
\usepackage{float}

%%% format and command for lab ans c and assembly

\newcommand{\HRule}{\rule{\linewidth}{0.4mm}} % Defines a new command for horizontal lines, change thickness here



%----------------------------------------------------------------------------------------
%	TITLE PAGE
%----------------------------------------------------------------------------------------


\newcommand{\CP}[4]{ \begin{titlepage} % Suppresses displaying the page number on the title page and the subsequent page counts as page 1
		%%%%  university logo%%
		\begin{figure}[H]
			\centering
			\includegraphics[scale=0.13]{tulogo.jpg}
		\end{figure}
		%%% end university logo

		\center % Centre everything on the page

		%------------------------------------------------
		%	Headings
		%------------------------------------------------

		\textsc{\huge Institute of Engineering \\ Central Campus,Pulchowk}\\[1.5cm] % Main heading such as the name of your university/college

		\textsc{\Large #1}\\[0.5cm] % Major heading such as course name

		\textsc{\large #2}\\[0.5cm] % Minor heading such as assignment no./ lab no.

		%------------------------------------------------
		%	Title
		%------------------------------------------------

		\HRule\\[0.4cm]

		{\Huge\bfseries #3}\\[0.4cm] % Title of your document

		\HRule\\[1.5cm]

		%------------------------------------------------
		%	Author(s)
		%------------------------------------------------
		\vfill\vfill
		\begin{minipage}{0.4\textwidth}
			\begin{flushleft}
				\large{
				\textbf{Submitted BY:}\\
				{\normalsize AMRIT PRASAD PHUYAL}\\ % NAME
				{\normalsize Roll: PULL074BEX004}} % Roll
			\end{flushleft}
		\end{minipage}
		~
		\begin{minipage}{0.4\textwidth}
			\begin{flushright}
				\large
				\textbf{Submitted To:}\\
				{ \normalsize{#4}\\ }% recepent's  Name 
				{\normalsize Department of Electronics and Computer Engineering}
			\end{flushright}
		\end{minipage}

		%------------------------------------------------
		%	Date
		%------------------------------------------------

		\vfill\vfill\vfill % Position the date 3/4 down the remaining page

		{\large\today} % Date, change the \today to a set date if you want to be precise

		\vfill % Push the date up 1/4 of the remaining page

	\end{titlepage}
} %%% cover page

%%% For CMD output %%%

%%%%%%%%% use  
%%% For CMD output %%%

%%%%%%%%% use  
%%% For CMD output %%%

%%%%%%%%% use  \include{CMD output.tex}
%%%%%%%%% use \CMD{###filename}{##Caption}
\usepackage{listings}

\usepackage{mdframed}
\usepackage{xcolor}
%\definecolor{codegreen}{rgb}{0,0.6,0}
%\definecolor{codegray}{rgb}{0.4,0.4,0.4}
%\definecolor{codepurple}{rgb}{0.58,0,0.82}
%\definecolor{blackcolour}{rgb}{0,0,0}


\definecolor{bluefront}{RGB}{10,214,255}
\definecolor{blueback}{RGB}{25,24,96}


\renewcommand{\lstlistlistingname}{List of CMD Outputs}
\renewcommand{\lstlistingname}{Output}


\lstdefinestyle{customa}{
    backgroundcolor=\color{blueback},
    %  keywordstyle=\color{magenta},
    %numberstyle=\tiny\color{codegray},
    %stringstyle=\color{codepurple},
    basicstyle=\ttfamily\scriptsize\color{bluefront},
    breakatwhitespace=false,
    breaklines=true,
    captionpos=b,
    %morekeywords={MOV,ADD,ADDC,ACALL,INC,DJNZ,AJMP,RET,END,ORG,RR,JNC,SUBB,JC,DEC},
    keepspaces=true,
    %numbers=left,
    %numbersep=5pt,
    showspaces=false,
    showstringspaces=false,
    showtabs=false,
    tabsize=4
}

\newcommand {\CMD}[2]{

    \begin{mdframed}[backgroundcolor=blueback,innerbottommargin=-2.3em,innertopmargin=-0.1em]
        \lstinputlisting[style=customa,caption={#2}]{#1}
    \end{mdframed}
}

%%% For CMD output %%%


%%%%%%%%% use \CMD{###filename}{##Caption}
\usepackage{listings}

\usepackage{mdframed}
\usepackage{xcolor}
%\definecolor{codegreen}{rgb}{0,0.6,0}
%\definecolor{codegray}{rgb}{0.4,0.4,0.4}
%\definecolor{codepurple}{rgb}{0.58,0,0.82}
%\definecolor{blackcolour}{rgb}{0,0,0}


\definecolor{bluefront}{RGB}{10,214,255}
\definecolor{blueback}{RGB}{25,24,96}


\renewcommand{\lstlistlistingname}{List of CMD Outputs}
\renewcommand{\lstlistingname}{Output}


\lstdefinestyle{customa}{
    backgroundcolor=\color{blueback},
    %  keywordstyle=\color{magenta},
    %numberstyle=\tiny\color{codegray},
    %stringstyle=\color{codepurple},
    basicstyle=\ttfamily\scriptsize\color{bluefront},
    breakatwhitespace=false,
    breaklines=true,
    captionpos=b,
    %morekeywords={MOV,ADD,ADDC,ACALL,INC,DJNZ,AJMP,RET,END,ORG,RR,JNC,SUBB,JC,DEC},
    keepspaces=true,
    %numbers=left,
    %numbersep=5pt,
    showspaces=false,
    showstringspaces=false,
    showtabs=false,
    tabsize=4
}

\newcommand {\CMD}[2]{

    \begin{mdframed}[backgroundcolor=blueback,innerbottommargin=-2.3em,innertopmargin=-0.1em]
        \lstinputlisting[style=customa,caption={#2}]{#1}
    \end{mdframed}
}

%%% For CMD output %%%


%%%%%%%%% use \CMD{###filename}{##Caption}
\usepackage{listings}

\usepackage{mdframed}
\usepackage{xcolor}
%\definecolor{codegreen}{rgb}{0,0.6,0}
%\definecolor{codegray}{rgb}{0.4,0.4,0.4}
%\definecolor{codepurple}{rgb}{0.58,0,0.82}
%\definecolor{blackcolour}{rgb}{0,0,0}


\definecolor{bluefront}{RGB}{10,214,255}
\definecolor{blueback}{RGB}{25,24,96}


\renewcommand{\lstlistlistingname}{List of CMD Outputs}
\renewcommand{\lstlistingname}{Output}


\lstdefinestyle{customa}{
    backgroundcolor=\color{blueback},
    %  keywordstyle=\color{magenta},
    %numberstyle=\tiny\color{codegray},
    %stringstyle=\color{codepurple},
    basicstyle=\ttfamily\scriptsize\color{bluefront},
    breakatwhitespace=false,
    breaklines=true,
    captionpos=b,
    %morekeywords={MOV,ADD,ADDC,ACALL,INC,DJNZ,AJMP,RET,END,ORG,RR,JNC,SUBB,JC,DEC},
    keepspaces=true,
    %numbers=left,
    %numbersep=5pt,
    showspaces=false,
    showstringspaces=false,
    showtabs=false,
    tabsize=4
}

\newcommand {\CMD}[2]{

    \begin{mdframed}[backgroundcolor=blueback,innerbottommargin=-2.3em,innertopmargin=-0.1em]
        \lstinputlisting[style=customa,caption={#2}]{#1}
    \end{mdframed}
}

%%% For CMD output %%%

 %%% Cmd OUTPUT blue background
\newcommand{\SYN}[1]{\quad Command used:-$\textbf{#1}$}

\begin{document}

%%%%  COver page 
\CP{Computer Network}{Lab \#2}{Study of Basic Networking Commands}
{SHARAD KUMAR GHIMIRE}
%%%%%%%%%%%%%%%%%%%%

\pagenumbering{gobble}
\renewcommand{\contentsname}{Table of Contents}
\tableofcontents

%\pagebreak
%\listoffigures
%\pagebreak
%\listoftables
\pagebreak
\lstlistoflistings
\pagebreak
\pagenumbering{arabic}

\section{Title} \textbf{Study of Basic Networking Commands}
%%%%%%%%%%%%%%%%%%%%%%%%%%%%
\section{Objective}
\begin{itemize}
    \item To be familiar with basic networking commands and their uses
\end{itemize}
%%%%%%%%%%%%%%%%%%%%%
\section{Requirement}
\begin{itemize}
    \item Computer with Internet Connectivity
\end{itemize}
%%%%%%%%%%%%%%%%%%%

\section {Procedure}
This lab session is code along so , we coded along with the instructor and
noted the output of some useful network command . Command line interface "CMD" for windows and for Linux "Terminal" is used .
Some code used are same and some differ as per the platforms. We study different command and their varient/parameters
some of them are :-


\begin{itemize}
    \item ipconfig
    \item ping
    \item getmac
    \item tracert
    \item arp
    \item hostname
    \item netstat
    \item route
\end{itemize}

\pagebreak
\section{Exercises}

%%%%%%%%%%%%%%%11111111111111
\begin{Q}
    {Explain the following commands briefly with their functions and few syntaxes.}
\end{Q}
\begin{enumerate}

    \item \textsc {\bfseries{ipconfig:}}
          As per the Documentation provided in Microsoft website it displays all TCP/IP current setup and refreshes
          DHCP and DNS settings. with the help of command prompt help statement all syntaxes and some of the function
          are listed below.

          \SYN{ipconfig /?}

          \CMD{Ipconfighelp.txt}{ipconfig Syntaxes and Uses}
          \pagebreak
    \item \textsc {\bfseries{ping:}}
          It is used along with url or Ip address. Source device will send packets,Internet Control Message Protocol (ICMP),
          to destination and waits for response.
          If the destination device responds it shows the round trip time in ms.Ping has various uses in troubleshoot Connectivity and name resolution.
          generally used to test whether the targeted domain or Ip has access to internet and if it loose any packet during transfer.
          All ping syntaxes ,parameters and its uses are listed below:

          \SYN{ping /?}

          \CMD{pinghelp.txt}{ping Syntaxes and Uses}


    \item \textsc {\bfseries{getmac:}}
          This command retrieves the MAC (Media Access Control) Address or Physical Address  of connected adapters .
          This command can also obtain the MAC address of Remote device/computer too.
          All getmac syntaxes ,parameters and its uses are listed below:

          \SYN{getmac /?}

          \CMD{./getmachelp.txt}{getmac Syntaxes and Uses}

    \item \textsc {\bfseries{tracert:}}
          It shows  different information about the path taken by packets to reach the destination.
          Each packets has TTL which decreases as it passes the Routers.
          All tracert syntaxes ,parameters and its uses are listed below:

          \SYN{tracert /?}

          \CMD{tracerthelp.txt}{tracert Syntaxes and Uses}

    \item \textsc {\bfseries {arp:}}
          ARP{Address Resolution Protocol} is used to pair MAC address with Ip address and save for future uses.
          This command has ability to display and modify the Cache of Address translation table if needed.
          All arp syntaxes ,parameters and its uses are listed below:

          \SYN{arp /?}

          \CMD{arphelp.txt}{arp Syntaxes and Uses}

    \item \textsc {\bfseries {hostname:}}
          Hostname is the device name in the network so this command displays the hostname of the device.
          it doesn't  have any other parameter other than help.

          \SYN{hostname /?}

          \CMD{hostnamehelp.txt}{hostname Syntaxes and Uses}


    \item \textsc {\bfseries {netstat:}}
          It shows the statistics about the connected network and devices.
          It informs about the current Working TCP/Ip  Connection including ports and addresses.
          It displays the open ports or ports listening (can establish connection).
          All netstat syntaxes ,parameters and its uses are listed below:
          \SYN{netstat /?}

          \CMD{netstathelp.txt}{netstat Syntaxes and Uses}

    \item \textsc {\bfseries{route:}}
          It has ability to print the content in IP routing tables and modify it if needed.
          All route syntaxes ,parameters and its uses are listed below:\\
          \SYN{route /?}

          \CMD{routehelp.txt}{route Syntaxes and Uses}

\end{enumerate}

%%%%%%%%%%%%%%%%%%%%%

%%%%%%%%%%%%%%%222222222222222222222
\begin{Q}
    {Note down the observation of each steps with necessary commands specified in activities B
        mentioned above and comment on it.}
\end{Q}
\begin{enumerate}
    \item \textbf{Using ipconfig:}
          \begin{enumerate}
              \item \textbf{ipconfig:}
                    Displays Windows ip configuration including IPv6/v4 address ,Subnet mask ,default gateway etc.
                    \CMD{ipconfig.txt}{ipconfig}

              \item \textbf{ipconfig/all:} Similar to $ipconfig$ but with additional information like Description, DHCP, Physical Address, Dns server etc.
                    \CMD{ipconfigall.txt}{ipconfig/all}


          \end{enumerate}

    \item \textbf{Using ping:}
          \begin{enumerate}
              \item \textbf{Obtain the IP address of your default gateway (refer qu. 1) and ping to that IP address.:}

                    My default gateway Ip address is 192.168.10.1 so command is  $ ping 192.168.10.1$.
                    It display the round trip time to and from my default gateway (Router). It also displays
                    no of data sent and received along with loss.

                    \CMD{pinggateway.txt}{ping Default Gateway}

              \item \textbf{ping “your ISP”:}
                    My ISP is Vianet so I should use $ ping$ $vianet.com.np $

                    \CMD{pingvianet.txt}{ping vianet.com.np}

              \item \textbf{ping google.com:}
                    \CMD{pinggoogle.txt}{ping google.com}

              \item \textbf{ping 103.5.150.3:}
                    \CMD {pingip.txt}{ ping 103.5.150.3}

          \end{enumerate}

    \item \textbf{Using getmac:}
          \begin{enumerate}
              \item \textbf{getmac:}
                    It displays MAC address (physical address) for all the available Network Adapters in the system .
                    It also display the mac address assigned for virtual adapter created by Virtual machine and VPN's in the system.


                    \CMD{getmac.txt}{getmac}

          \end{enumerate}

    \item \textbf{Using tracert:}
          \begin{enumerate}
              \item \textbf{Obtain the IP address of your default gateway (refer qu. 1) and use tracert to that IP
                        address.:} \\
                    My default gateway ip(Router)  is 192.168.10.1 so below is the output of the tracert command on Router.

                    \CMD{tracert.txt}{tracert Deafault Gateway}

              \item \textbf{tracert “your ISP”:}
                    \CMD{traceisp.txt}{tracert ISP}

              \item \textbf{tracert google.com:}
                    \CMD{tracegoogle.txt}{tracert google.com}

              \item \textbf{tracert 103.5.150.3:}

                    \CMD{traceip.txt}{tracert 103.5.150.3 }


          \end{enumerate}

    \item \textbf{Using arp:}
          \begin{enumerate}
              \item \textbf{arp -a:} Display ARP table

                    \CMD{arpa.txt}{arp -a}


              \item \textbf{Use ping to any another device within your network such as another computer or laptop
                        or mobile phone or tablet etc. and again use arp –a (if there are multiple devices in your
                        network you can ping one by one by observing the output of arp -a after each ping)}\\
                    I am pinging my phone with Ip asssigned 192.168.10.102 and physical address d8:0b:9a:3a:37:f4.
                    I also pinged another device with 192.168.10.104 and  MAC address ec:35:86:7e:4e:30.
                    In ARP table both the devices are visible with their corresponding MAC address
                    just after my default gateway 192.168.10.1 with type Dynanic\\
                    **** Only changes from arp -a are included in output

                    \CMD{arpaping.txt}{arp -a after pinging another device}
          \end{enumerate}

    \item \textbf{Using hostname:}
          \begin{enumerate}
              \item \textbf{hostname:} Displays my System Hostname.
                    \CMD{hostname.txt}{hostname}


          \end{enumerate}

    \item \textbf{Using netstat:}
          \begin{enumerate}
              \item \textbf{netstat -a:} Display all connection and listening Ports\\
                    **** some entries are deleted  to meet the memory requirement for TEX.
                    \CMD{netstata.txt}{netstat -a}

              \item \textbf{netstat -e:}
                    Display ethernet statistics like Size of packet received and send in Bytes,
                    errors uni and non- unicast packets etc.
                    \CMD{netstate.txt}{netstat -e}

              \item \textbf{netstat -r:}
                    Display Interface list along with IPv6 and IPv4 Routing table.
                    \CMD{netstatr.txt}{netstat -r}

          \end{enumerate}

    \item \textbf{Using route:}
          \begin{enumerate}
              \item \textbf{route print:}
                    Dispaly interface list along with IPv4 and IPv6 Routing Table.

                    \CMD{routep.txt}{route print}

              \item \textbf{route print -4:}
                    Display only IPv4 Routing Table

                    \CMD{routep4.txt}{route print -4}

              \item \textbf{route print -6:}
                    Display only IPv6 Routing Table
                    \CMD{routep6.txt}{route print -6}

          \end{enumerate}

\end{enumerate}


%%%%%%%%%%%%%%%%%%%%%
\pagebreak

%%%%%%%%%%%%%%%333333333333333333333333
\begin{Q}
    {What is the actual IP address of your computer? Also find the Public IP address that is being
        used for your computer’s Internet connectivity. Note down both the IP addresses.}
\end{Q}
\begin{A}
    {
        My actual ip of my computer is provided in ipconfig output i.e 192.168.10.106

        \CMD{pcip.txt}{IP address of my PC}

        There are different methods and sites to find Public IP used for internet connectivity .\\

        I choose  nslookup  command  and used \textit{nslookup myip.opendns.com resolver1.opendns.com}
        to generate following output . So, my public ip is 43.245.87.194 and confirmed with external sites too.

        \CMD{pubip.txt}{Public IP address}

    }
\end{A}
%%%%%%%%%%%%%%%%%%%%%



\section{Conclusion}
In this LAB -2 we get familiar with different network commands and their uses in Linux and windows platform.
We learned 8 commands with majority of them having some additional argument.
We learned to ping the computer or device in and outside the network. We learned to extract IP and MAC addresses of connected adapters .
We also learned about ping , trace routing,ip configuration and many other tools useful for troubleshooting and connectivity.
We familiarized ourself with public and private ip and methods to find them.



\end{document}