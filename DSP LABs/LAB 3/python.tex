


% %%%%%%%%% use  


% %%%%%%%%% use  


% %%%%%%%%% use  


% %%%%%%%%% use  \input{python.tex}
% %%%%%%%%% use \PY{###filename}{##Caption}
% \usepackage{listings}

% \usepackage{mdframed}
% \usepackage{xcolor}
% \definecolor{codegreen}{rgb}{0.1,0.9,0.2}
% \definecolor{num}{rgb}{0.1,0.2,0.6}
% \definecolor{keyword}{rgb}{0.1,0.1,0.99}
% \definecolor{blackcolour}{rgb}{0,0,0}
% \definecolor{string}{rgb}{0.78,0.2,0.32}


% \definecolor{front}{RGB}{8,39,20}
% \definecolor{back}{RGB}{10,45,98}

% % \renewcommand{\lstlistlistingname}{List of Python codes}
% % \renewcommand{\lstlistingname}{Code}


\lstdefinestyle{custopy}{
    language=Python,
    keywordstyle=\color{keyword},
    numberstyle=\tiny\color{num},
    stringstyle=\color{string},
    commentstyle=\color{codegreen},
    basicstyle=\ttfamily\footnotesize\color{front},
    breakatwhitespace=false,
    breaklines=true,
    captionpos=b,
    morekeywords={np,plt,stem,xlabel,ylabel,array,arange,zeros,transpose,savefig,title,show},
    keepspaces=true,
    numbers=left,
    numbersep=15pt,
    showspaces=false,
    showstringspaces=false,
    showtabs=false,
    tabsize=4
}

\newcommand {\PY}[2]{

    \begin{mdframed}[innerbottommargin=-2.3em,innertopmargin=-0.1em]
        \lstinputlisting[style=custopy,caption={#2}]{#1}
    \end{mdframed}
}

% %%%%%%%%% use \PY{###filename}{##Caption}
% \usepackage{listings}

% \usepackage{mdframed}
% \usepackage{xcolor}
% \definecolor{codegreen}{rgb}{0.1,0.9,0.2}
% \definecolor{num}{rgb}{0.1,0.2,0.6}
% \definecolor{keyword}{rgb}{0.1,0.1,0.99}
% \definecolor{blackcolour}{rgb}{0,0,0}
% \definecolor{string}{rgb}{0.78,0.2,0.32}


% \definecolor{front}{RGB}{8,39,20}
% \definecolor{back}{RGB}{10,45,98}

% % \renewcommand{\lstlistlistingname}{List of Python codes}
% % \renewcommand{\lstlistingname}{Code}


\lstdefinestyle{custopy}{
    language=Python,
    keywordstyle=\color{keyword},
    numberstyle=\tiny\color{num},
    stringstyle=\color{string},
    commentstyle=\color{codegreen},
    basicstyle=\ttfamily\footnotesize\color{front},
    breakatwhitespace=false,
    breaklines=true,
    captionpos=b,
    morekeywords={np,plt,stem,xlabel,ylabel,array,arange,zeros,transpose,savefig,title,show},
    keepspaces=true,
    numbers=left,
    numbersep=15pt,
    showspaces=false,
    showstringspaces=false,
    showtabs=false,
    tabsize=4
}

\newcommand {\PY}[2]{

    \begin{mdframed}[innerbottommargin=-2.3em,innertopmargin=-0.1em]
        \lstinputlisting[style=custopy,caption={#2}]{#1}
    \end{mdframed}
}

% %%%%%%%%% use \PY{###filename}{##Caption}
% \usepackage{listings}

% \usepackage{mdframed}
% \usepackage{xcolor}
% \definecolor{codegreen}{rgb}{0.1,0.9,0.2}
% \definecolor{num}{rgb}{0.1,0.2,0.6}
% \definecolor{keyword}{rgb}{0.1,0.1,0.99}
% \definecolor{blackcolour}{rgb}{0,0,0}
% \definecolor{string}{rgb}{0.78,0.2,0.32}


% \definecolor{front}{RGB}{8,39,20}
% \definecolor{back}{RGB}{10,45,98}

% % \renewcommand{\lstlistlistingname}{List of Python codes}
% % \renewcommand{\lstlistingname}{Code}


\lstdefinestyle{custopy}{
    language=Python,
    keywordstyle=\color{keyword},
    numberstyle=\tiny\color{num},
    stringstyle=\color{string},
    commentstyle=\color{codegreen},
    basicstyle=\ttfamily\footnotesize\color{front},
    breakatwhitespace=false,
    breaklines=true,
    captionpos=b,
    morekeywords={np,plt,stem,xlabel,ylabel,array,arange,zeros,transpose,savefig,title,show},
    keepspaces=true,
    numbers=left,
    numbersep=15pt,
    showspaces=false,
    showstringspaces=false,
    showtabs=false,
    tabsize=4
}

\newcommand {\PY}[2]{

    \begin{mdframed}[innerbottommargin=-2.3em,innertopmargin=-0.1em]
        \lstinputlisting[style=custopy,caption={#2}]{#1}
    \end{mdframed}
}

% %%%%%%%%% use \PY{###filename}{##Caption}
% \usepackage{listings}

% \usepackage{mdframed}
% \usepackage{xcolor}
% \definecolor{codegreen}{rgb}{0.1,0.9,0.2}
% \definecolor{num}{rgb}{0.1,0.2,0.6}
% \definecolor{keyword}{rgb}{0.1,0.1,0.99}
% \definecolor{blackcolour}{rgb}{0,0,0}
% \definecolor{string}{rgb}{0.78,0.2,0.32}


% \definecolor{front}{RGB}{8,39,20}
% \definecolor{back}{RGB}{10,45,98}

% % \renewcommand{\lstlistlistingname}{List of Python codes}
% % \renewcommand{\lstlistingname}{Code}


\lstdefinestyle{custopy}{
    language=Python,
    keywordstyle=\color{keyword},
    numberstyle=\tiny\color{num},
    stringstyle=\color{string},
    commentstyle=\color{codegreen},
    basicstyle=\ttfamily\footnotesize\color{front},
    breakatwhitespace=false,
    breaklines=true,
    captionpos=b,
    morekeywords={np,plt,stem,xlabel,ylabel,array,arange,zeros,transpose,savefig,title,show},
    keepspaces=true,
    numbers=left,
    numbersep=15pt,
    showspaces=false,
    showstringspaces=false,
    showtabs=false,
    tabsize=4
}

\newcommand {\PY}[2]{

    \begin{mdframed}[innerbottommargin=-2.3em,innertopmargin=-0.1em]
        \lstinputlisting[style=custopy,caption={#2}]{#1}
    \end{mdframed}
}
