\documentclass[a4paper,11pt]{article}
\usepackage{geometry}
 \geometry{
 a4paper,
 total={170mm,257mm},
 left=20mm,
 top=20mm,
 }

 \usepackage{enumerate}
 \usepackage{amsmath}
 \usepackage{siunitx}
 \usepackage{multirow}
\usepackage{colortbl}
 \usepackage{hhline}

 \usepackage{lipsum}  %%% Lorem ipsum

\setlength{\headheight}{30.0pt}
\setlength{\footskip}{20pt}


\usepackage{hyperref}
\hypersetup{
    colorlinks=True,
    linkcolor={blue!20!black},
    filecolor=magenta,      
    urlcolor=cyan,
}



 \usepackage[export]{adjustbox}
\usepackage[english]{babel}
\usepackage[utf8]{inputenc}
\usepackage{fancyhdr}
\usepackage{multicol}

\pagestyle{fancy}
\fancyhf{}
\rhead{\textit{Pul074BEX004}}
\lhead{\textit{Amrit Prasad Phuyal}}
\rfoot{\thepage}


\usepackage{mathpazo} % Palatino font
\usepackage{graphicx}
\usepackage{float}


%%%%%% include  Titles.%%%% use \input{./CP}%%%
%%%use """"""""    \CP{}{}{}{}   """" %%%% and 4 argument to craete Title page 
%%%%%%%%%%%%%%%%%%%%%%%%%%%%%%%%%%%%%%%%%%%%%%%%%%%%%%%%%%%%%%%%%
%%%argument number
%% 1=major header ## Course name 
%% 2=minor4 heading ## lab/assignmet no
%% 3=Title  ## Assignment or Lab title
%% 4=submitted to::## input receiver Name"
%%%%%%%%%%%%%%%%%%%%%%%%%%%%%%%%%%%%%%%%%%%%%%%%%%%%%%%%%%%%%%%%%


\usepackage{mathpazo} % Palatino font
\usepackage{graphicx}
\usepackage{float}

%%% format and command for lab ans c and assembly

\newcommand{\HRule}{\rule{\linewidth}{0.4mm}} % Defines a new command for horizontal lines, change thickness here



%----------------------------------------------------------------------------------------
%	TITLE PAGE
%----------------------------------------------------------------------------------------


\newcommand{\CP}[4]{ \begin{titlepage} % Suppresses displaying the page number on the title page and the subsequent page counts as page 1
		%%%%  university logo%%
		\begin{figure}[H]
			\centering
			\includegraphics[scale=0.13]{tulogo.jpg}
		\end{figure}
		%%% end university logo

		\center % Centre everything on the page

		%------------------------------------------------
		%	Headings
		%------------------------------------------------

		\textsc{\huge Institute of Engineering \\ Central Campus,Pulchowk}\\[1.5cm] % Main heading such as the name of your university/college

		\textsc{\Large #1}\\[0.5cm] % Major heading such as course name

		\textsc{\large #2}\\[0.5cm] % Minor heading such as assignment no./ lab no.

		%------------------------------------------------
		%	Title
		%------------------------------------------------

		\HRule\\[0.4cm]

		{\Huge\bfseries #3}\\[0.4cm] % Title of your document

		\HRule\\[1.5cm]

		%------------------------------------------------
		%	Author(s)
		%------------------------------------------------
		\vfill\vfill
		\begin{minipage}{0.4\textwidth}
			\begin{flushleft}
				\large{
				\textbf{Submitted BY:}\\
				{\normalsize AMRIT PRASAD PHUYAL}\\ % NAME
				{\normalsize Roll: PULL074BEX004}} % Roll
			\end{flushleft}
		\end{minipage}
		~
		\begin{minipage}{0.4\textwidth}
			\begin{flushright}
				\large
				\textbf{Submitted To:}\\
				{ \normalsize{#4}\\ }% recepent's  Name 
				{\normalsize Department of Electronics and Computer Engineering}
			\end{flushright}
		\end{minipage}

		%------------------------------------------------
		%	Date
		%------------------------------------------------

		\vfill\vfill\vfill % Position the date 3/4 down the remaining page

		{\large\today} % Date, change the \today to a set date if you want to be precise

		\vfill % Push the date up 1/4 of the remaining page

	\end{titlepage}
} %%% cover page



%%%%%%%%% use  


%%%%%%%%% use  


%%%%%%%%% use  \input{Matlab.tex}
%%%%%%%%% use \MAT{###filename}{##Caption}
\usepackage{listings}

\usepackage{mdframed}
\usepackage{xcolor}
\definecolor{codegreen}{rgb}{0.1,0.9,0.2}
\definecolor{num}{rgb}{0.1,0.2,0.6}
\definecolor{keyword}{rgb}{0.1,0.1,0.99}
\definecolor{blackcolour}{rgb}{0,0,0}
\definecolor{string}{rgb}{0.9078,0.2,0.32}


\definecolor{front}{RGB}{8,39,20}
\definecolor{back}{RGB}{10,45,98}

% \renewcommand{\lstlistlistingname}{List of MATLAB codes}
% \renewcommand{\lstlistingname}{Code}

\renewcommand{\lstlistlistingname}{List of Matlab codes}
\renewcommand{\lstlistingname}{Code}


\lstdefinestyle{customa}{
    language=Matlab,
    % morekeywords={zeros,length},
    keywordstyle=\color{keyword},
    numberstyle=\tiny\color{num},
    stringstyle=\color{string},
    commentstyle=\color{codegreen},
    basicstyle=\ttfamily\footnotesize\color{front},
    breakatwhitespace=false,
    breaklines=true,
    captionpos=b,
    keepspaces=true,
    numbers=left,
    numbersep=15pt,
    showspaces=false,
    showstringspaces=false,
    showtabs=false,
    tabsize=4
}

\newcommand {\MAT}[2]{

    \begin{mdframed}[innerbottommargin=-2.3em,innertopmargin=-0.1em]
        \lstinputlisting[style=customa,caption={#2}]{#1}
    \end{mdframed}
}


% backgroundcolor=blueback,

%   innerbottommargin=-2.3em,innertopmargin=-0.1em,
%%     [outermargin =+1cm,]
%%%%%%%%% use \MAT{###filename}{##Caption}
\usepackage{listings}

\usepackage{mdframed}
\usepackage{xcolor}
\definecolor{codegreen}{rgb}{0.1,0.9,0.2}
\definecolor{num}{rgb}{0.1,0.2,0.6}
\definecolor{keyword}{rgb}{0.1,0.1,0.99}
\definecolor{blackcolour}{rgb}{0,0,0}
\definecolor{string}{rgb}{0.9078,0.2,0.32}


\definecolor{front}{RGB}{8,39,20}
\definecolor{back}{RGB}{10,45,98}

% \renewcommand{\lstlistlistingname}{List of MATLAB codes}
% \renewcommand{\lstlistingname}{Code}

\renewcommand{\lstlistlistingname}{List of Matlab codes}
\renewcommand{\lstlistingname}{Code}


\lstdefinestyle{customa}{
    language=Matlab,
    % morekeywords={zeros,length},
    keywordstyle=\color{keyword},
    numberstyle=\tiny\color{num},
    stringstyle=\color{string},
    commentstyle=\color{codegreen},
    basicstyle=\ttfamily\footnotesize\color{front},
    breakatwhitespace=false,
    breaklines=true,
    captionpos=b,
    keepspaces=true,
    numbers=left,
    numbersep=15pt,
    showspaces=false,
    showstringspaces=false,
    showtabs=false,
    tabsize=4
}

\newcommand {\MAT}[2]{

    \begin{mdframed}[innerbottommargin=-2.3em,innertopmargin=-0.1em]
        \lstinputlisting[style=customa,caption={#2}]{#1}
    \end{mdframed}
}


% backgroundcolor=blueback,

%   innerbottommargin=-2.3em,innertopmargin=-0.1em,
%%     [outermargin =+1cm,]
%%%%%%%%% use \MAT{###filename}{##Caption}
\usepackage{listings}

\usepackage{mdframed}
\usepackage{xcolor}
\definecolor{codegreen}{rgb}{0.1,0.9,0.2}
\definecolor{num}{rgb}{0.1,0.2,0.6}
\definecolor{keyword}{rgb}{0.1,0.1,0.99}
\definecolor{blackcolour}{rgb}{0,0,0}
\definecolor{string}{rgb}{0.9078,0.2,0.32}


\definecolor{front}{RGB}{8,39,20}
\definecolor{back}{RGB}{10,45,98}

% \renewcommand{\lstlistlistingname}{List of MATLAB codes}
% \renewcommand{\lstlistingname}{Code}

\renewcommand{\lstlistlistingname}{List of Matlab codes}
\renewcommand{\lstlistingname}{Code}


\lstdefinestyle{customa}{
    language=Matlab,
    % morekeywords={zeros,length},
    keywordstyle=\color{keyword},
    numberstyle=\tiny\color{num},
    stringstyle=\color{string},
    commentstyle=\color{codegreen},
    basicstyle=\ttfamily\footnotesize\color{front},
    breakatwhitespace=false,
    breaklines=true,
    captionpos=b,
    keepspaces=true,
    numbers=left,
    numbersep=15pt,
    showspaces=false,
    showstringspaces=false,
    showtabs=false,
    tabsize=4
}

\newcommand {\MAT}[2]{

    \begin{mdframed}[innerbottommargin=-2.3em,innertopmargin=-0.1em]
        \lstinputlisting[style=customa,caption={#2}]{#1}
    \end{mdframed}
}


% backgroundcolor=blueback,

%   innerbottommargin=-2.3em,innertopmargin=-0.1em,
%%     [outermargin =+1cm,] %%% Matlab code



% %%%%%%%%% use  


% %%%%%%%%% use  


% %%%%%%%%% use  \input{python.tex}
% %%%%%%%%% use \PY{###filename}{##Caption}
% \usepackage{listings}

% \usepackage{mdframed}
% \usepackage{xcolor}
% \definecolor{codegreen}{rgb}{0.1,0.9,0.2}
% \definecolor{num}{rgb}{0.1,0.2,0.6}
% \definecolor{keyword}{rgb}{0.1,0.1,0.99}
% \definecolor{blackcolour}{rgb}{0,0,0}
% \definecolor{string}{rgb}{0.78,0.2,0.32}


% \definecolor{front}{RGB}{8,39,20}
% \definecolor{back}{RGB}{10,45,98}

% % \renewcommand{\lstlistlistingname}{List of Python codes}
% % \renewcommand{\lstlistingname}{Code}


\lstdefinestyle{custopy}{
    language=Python,
    keywordstyle=\color{keyword},
    numberstyle=\tiny\color{num},
    stringstyle=\color{string},
    commentstyle=\color{codegreen},
    basicstyle=\ttfamily\footnotesize\color{front},
    breakatwhitespace=false,
    breaklines=true,
    captionpos=b,
    morekeywords={np,plt,stem,xlabel,ylabel,array,arange,zeros,transpose,savefig,title,show},
    keepspaces=true,
    numbers=left,
    numbersep=15pt,
    showspaces=false,
    showstringspaces=false,
    showtabs=false,
    tabsize=4
}

\newcommand {\PY}[2]{

    \begin{mdframed}[innerbottommargin=-2.3em,innertopmargin=-0.1em]
        \lstinputlisting[style=custopy,caption={#2}]{#1}
    \end{mdframed}
}

% %%%%%%%%% use \PY{###filename}{##Caption}
% \usepackage{listings}

% \usepackage{mdframed}
% \usepackage{xcolor}
% \definecolor{codegreen}{rgb}{0.1,0.9,0.2}
% \definecolor{num}{rgb}{0.1,0.2,0.6}
% \definecolor{keyword}{rgb}{0.1,0.1,0.99}
% \definecolor{blackcolour}{rgb}{0,0,0}
% \definecolor{string}{rgb}{0.78,0.2,0.32}


% \definecolor{front}{RGB}{8,39,20}
% \definecolor{back}{RGB}{10,45,98}

% % \renewcommand{\lstlistlistingname}{List of Python codes}
% % \renewcommand{\lstlistingname}{Code}


\lstdefinestyle{custopy}{
    language=Python,
    keywordstyle=\color{keyword},
    numberstyle=\tiny\color{num},
    stringstyle=\color{string},
    commentstyle=\color{codegreen},
    basicstyle=\ttfamily\footnotesize\color{front},
    breakatwhitespace=false,
    breaklines=true,
    captionpos=b,
    morekeywords={np,plt,stem,xlabel,ylabel,array,arange,zeros,transpose,savefig,title,show},
    keepspaces=true,
    numbers=left,
    numbersep=15pt,
    showspaces=false,
    showstringspaces=false,
    showtabs=false,
    tabsize=4
}

\newcommand {\PY}[2]{

    \begin{mdframed}[innerbottommargin=-2.3em,innertopmargin=-0.1em]
        \lstinputlisting[style=custopy,caption={#2}]{#1}
    \end{mdframed}
}

% %%%%%%%%% use \PY{###filename}{##Caption}
% \usepackage{listings}

% \usepackage{mdframed}
% \usepackage{xcolor}
% \definecolor{codegreen}{rgb}{0.1,0.9,0.2}
% \definecolor{num}{rgb}{0.1,0.2,0.6}
% \definecolor{keyword}{rgb}{0.1,0.1,0.99}
% \definecolor{blackcolour}{rgb}{0,0,0}
% \definecolor{string}{rgb}{0.78,0.2,0.32}


% \definecolor{front}{RGB}{8,39,20}
% \definecolor{back}{RGB}{10,45,98}

% % \renewcommand{\lstlistlistingname}{List of Python codes}
% % \renewcommand{\lstlistingname}{Code}


\lstdefinestyle{custopy}{
    language=Python,
    keywordstyle=\color{keyword},
    numberstyle=\tiny\color{num},
    stringstyle=\color{string},
    commentstyle=\color{codegreen},
    basicstyle=\ttfamily\footnotesize\color{front},
    breakatwhitespace=false,
    breaklines=true,
    captionpos=b,
    morekeywords={np,plt,stem,xlabel,ylabel,array,arange,zeros,transpose,savefig,title,show},
    keepspaces=true,
    numbers=left,
    numbersep=15pt,
    showspaces=false,
    showstringspaces=false,
    showtabs=false,
    tabsize=4
}

\newcommand {\PY}[2]{

    \begin{mdframed}[innerbottommargin=-2.3em,innertopmargin=-0.1em]
        \lstinputlisting[style=custopy,caption={#2}]{#1}
    \end{mdframed}
}


\newcommand\ddfrac[2]{\frac{\displaystyle #1}{\displaystyle #2}} 



%%%%%%%%%%%%%%%%%%%%%for matlab observation #1 fig name #2 Caption
\newcommand{\mobs}[2]{
    \begin{figure}[H]
        \centering
        \includegraphics[width=1.07\linewidth]{./FIG/#1.eps}
        \caption{#2}
    \end{figure}
   
}

 %%%%%%%%%%% for Python observation #1 fig name #2 Caption

\newcommand{\pobs}[2]{
    \begin{figure}[H]
        \centering
        \includegraphics[width=1.07\linewidth]{./FIG/#1.eps}
        \caption{#2}
    \end{figure}
   
}




\begin{document}


%%%%  COver page 
\CP{Digital Signal Processing}{Lab \#3}{Convolution}
{Anila  Kansakar}
%%%%%%%%%%%%%%%%%%%%

\pagenumbering{gobble}
\renewcommand{\contentsname}{Table of Contents}
\tableofcontents

\pagebreak
%\listoffigures
% \pagebreak
% \vspace{5em}
\lstlistoflistings
\vspace{10em}
% \pagebreak
\listoffigures
\pagebreak
\pagenumbering{arabic}

%%%%%%%%%%%%%%%%%%%%%%%%%%%%%%%%%%%%%%%%%%%%%%
\section{Title} {\large Convolution}
%%%%%%%%%%%%%%%%%%%%%%%%%%%%
\section{Objective}
\begin {itemize}
\item To be able to perform convolution of two given signal using basic formula.
\item To be able to perform convolution of two given signals using Matlab function.
\item To be able to perform convolution of two given signals using Python package.

\end{itemize}
%%%%%%%%%%%%%%%%%%%%%


%Theory
\section{Theory}

\subsection {Matlab Function Used}
\begin{verbatim}
    Conv,Sinc etc.
\end{verbatim}

\subsection {Python Package Used}
\begin{verbatim}
    NumPy and Matlplotlib
\end{verbatim}

\subsection{Background}
\subsubsection{Convolution Sum}
The output of any Linear Time Invariant (LTI) system is some sort of operation between
input and system response; the operation is nothing but convolution, denoted by symbol \{*\}, and defined as

\begin{equation*}
    y(t)=x(t)*h(t)=\int_{-\infty}^{\infty}x(\tau)h(t-\tau)d\tau \text{   - - - For Continous Time}
\end{equation*}

\begin{equation*}
    y[n]=x[n] * h[n]=\int_{k=-\infty}^{\infty}x[k]h[n-k]  \text{   - - - For Discrete Time }
\end{equation*}
\\
For a causal LTI system, the convolution sum is given by,
\begin{equation*}
    y[n]=\int_{k=0}^{n}x[k]h[n-k]
\end{equation*}
\\
The process of computing the convolution between x[k] and h[k] involves the following four
steps:

\begin{enumerate}
    \item \textbf{Folding}: Fold $h[k]$ about $k=0$ to obtain $h[-k]$.
          \MAT{./CODES/sigfold.m}{Matlab Function for Folding Signal}
    \item \textbf{Shifting}: Shift $h[-k]$ by $n_0$ to the right (left if $n_o$ is positive (negative), to obtain $h[n_o-k]$.
          \MAT{./CODES/sigshift.m}{Matlab Function for Shifting Signal}
    \item \textbf{Multiplication}: Multiply $x[k]$ by $h[n_o-k]$ to obtain the product sequence $V_{n_o}[k]$ where, $V_{n_o}[k]=x[k].h[n_o-k]$.
          \MAT{./CODES/sigmulti.m}{Matlab Function for Multiplication of Signal}
    \item \textbf{Summation}: Sum all the values of the product sequence $V_{n_o}$ to obtain the value of  the output at times $n=n_o$.
\end{enumerate}

\subsubsection{Example} Find convolution of x[n] and h[n], where
\begin{verbatim}
x=[1,0,-1,1,2,1]
n1=[-2,-1,0,1,2,3]
h=[1,1,1,1,1]
n2=[0,1,2,3,4]
\end{verbatim}

\begin{enumerate}
    \item \textbf{Graphical Method}
          \MAT{./CODES/graphical.m}{Matlab code for Convolution using Graphical method }
          \mobs{graphical}{Convolution using Graphical method}
    \item \textbf{Tabular Method}
          \MAT{./CODES/Tabular.m}{Matlab code for Convolution using Tabular method }
          \mobs{Tabular}{Convolution using Tabular method}
    \item \textbf{\textit{conv} Function}
          \MAT{./CODES/conv_func.m}{Matlab code for Convolution using $conv$ function }
          \mobs{conv_func}{Convolution using $conv$ function}

\end{enumerate}

\section {Lab Problems}

%%%%%%%%%%%%Problem 1

\subsection{Problem 1}
\subsection*{Find the convolution result of the following signal using basic convolution formula :}

\begin{verbatim}
    X1(n1)=[1,1,1,1,1]
    n1=[-2,-1,0,1,2]
    X2(n2)=[1,0,0,0,0,0,0,0,0,0]
    n2=[-4,-3,-2,-1,0,1,2,3,4,5]
    X2 is a periodic signal.
    Y2=X1*X2
\end{verbatim}

%%%%%%%%%%%%% Matlab basic convolution formula
\subsubsection*{Matlab}

\MAT{./CODES/p1.m}{Matlab code for Convolution using basic convolution formula}
\mobs{p1}{Convolution using basic formula  \{Matlab\} }

% %%%%%%%%%% Python 
\subsubsection*{Python}

\PY{./CODES/p1p.py}{Python code for convolution using basic convolution formula  }
\pobs{p1p}{Convolution using basic convolution formula \{Python\} }


%%%%%%%%%%%Problem 2

\subsection{Problem 2}
\subsection*{Find the convolution of using \texttt{\{conv\}} function for matlab and \texttt{\{convolve\}} functon from numpy package for python}



{\begin{enumerate}[a.]
        %%%%%%%%%%%Problem 2a 2a 2a
        \item {\begin{equation*}
                  x[n]= \begin{cases}
                      \ddfrac{1}{3}n \quad \text{for} \quad 0\leq n \leq 6 \\
                      0 \quad \text{else}
                  \end{cases}
                  \quad\text{and,}\quad
                  h[n]= \begin{cases}
                      1 \quad \text{for} \quad -2\leq n \leq 2 \\
                      0 \quad \text{else}
                  \end{cases}
              \end{equation*} }


              %%%%%%%%%%%%% Matlab 
              \subsubsection*{Matlab}
              \MAT{./CODES/p2a.m}{Matlab code for DT Convolution using \texttt{\{conv\}} function}
              \mobs{p2a}{DT Convolution using \texttt{\{conv\}} function \{Matlab\} }


              % %%%%%%%%%% Python 
              \subsubsection*{Python}
              \PY{./CODES/p2ap.py}{Python code for DT Convolution using \texttt{\{convolve\}} function}
              \pobs{p2ap}{DT Convolution using \texttt{\{convolve\}} function \{Python\} }

              %%%%%%%%%%%Problem 2b 2b 2b
        \item {$$
                  x(t)=u(t)\quad \text{and,} \quad
                  h(t)=e^{-at}u(t), \text{ where }a>0
              $$}

              % %%%%%%%%%%%%% Matlab 

              \subsubsection*{Matlab}
              \MAT{./CODES/p2b.m}{Matlab code for CT Convolution using \texttt{\{conv\}} function}
              \mobs{p2b}{CT Convolution using \texttt{\{conv\}} function  \{Matlab\} }

              % %%%%%%%%%% Python 
              \subsubsection*{Python}

              \PY{./CODES/p2bp.py}{Python code for CT Convolution using \texttt{\{convolve\}} function}
              \pobs{p2bp}{CT Convolution using \texttt{\{convolve\}} function  \{Python\} }



    \end{enumerate}}

%%%%%%%%%%%%Problem 3

\subsection{Problem 3}
\subsection*{Consider two discrete time sequences $x[n]$ and $h[n]$ given by
\begin{equation*}
    x[n]=\begin{cases}
        1 \quad for \quad 0\leq n \leq 4 \\
        0 \quad elsewhere
    \end{cases}
    \quad \textbf{and,} \quad h[n]=\begin{cases}
        2^n \quad for \quad 0\leq n \leq 6 \\
        0 \quad elsewhere
    \end{cases}
\end{equation*}
{\begin{enumerate}[a.]
    \item Find the response of the LTI system with impulse response $h[n]$ to input $x[n]$.
    \item Plot the signals and comment on the result.
\end{enumerate}} }

% %%%%%%%%%%%%% Matlab 

\subsubsection*{Matlab}
\MAT{./CODES/p3.m}{Matlab code to find the response of the LTI system  }
\mobs{p3}{Response of LTI system  \{Matlab\} }

% %%%%%%%%%% Python 
\subsubsection*{Python}

\PY{./CODES/p3p.py}{Python code to find the response of the LTI system  }
\pobs{p3p}{ Response of LTI system  \{Python\} }



%%%%%%%%%% Problem 4

\subsection{Problem 4}
\subsection*{If the impulse response $h(t)$ of a LTI system is given by \texttt{sinc} function and input signal $x(t)$ is a rectangular wave given as,
    $$h(t)=\ddfrac{2\tau}{T_p}sinc\left(\ddfrac{2\tau t}{T_p}\right) \quad \text{and,} \quad x(t)=\begin{cases}
            1 \quad \text{for} \quad 1\leq t \leq 100 \\
            0 \quad \text{elsewhere}
        \end{cases}$$
    Find output of the system for different values of $\tau$. Comment on the result.}

% %%%%%%%%%%%%% Matlab 
\subsubsection*{Matlab}

\MAT{./CODES/p4.m}{Matlab code to find output of the LTI system where sinc function as impulse response  }
\mobs{p4}{Output of the LTI system where sinc function as impulse response  \{Matlab\} }

% %%%%%%%%%% Python 
\subsubsection*{Python}

\PY{./CODES/p4p.py}{ Python code to find output of the LTI system where sinc function as impulse response  }
\pobs{p4p}{ Output of the LTI system where sinc function as impulse response  \{Python\} }



%Discussion and Conclusion
\section{Discussion and Conclusion}
In this Lab we familiarize ourself with Matlab and Python Programming (Numpy and Matplotlib package)  with basic of Convolution with Continuous time and Discrete time signals. We explored Graphical, Tabular and builtin functions of Matlab and Python (numpy package). We also used the concept of convolution and convolution theorem to find the response of the LTI system. We experiment with \texttt{\{conv\}} function of matlab and \texttt{\{convolve\}} function of numpy package. We used the concept of DT and CT signals to find the response of the LTI system. We used the concept of sinc function to find the output of the LTI system. To visualize result in python we used the concept of \texttt{\{plot\}} function of matplotlib package.

\end{document}