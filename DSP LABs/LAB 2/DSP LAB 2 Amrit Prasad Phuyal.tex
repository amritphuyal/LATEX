\documentclass[a4paper,11pt]{article}
\usepackage{geometry}
 \geometry{
 a4paper,
 total={170mm,257mm},
 left=20mm,
 top=20mm,
 }


 \usepackage{amsmath}
 \usepackage{siunitx}
 \usepackage{multirow}
\usepackage{colortbl}
 \usepackage{hhline}

 \usepackage{lipsum}  %%% Lorem ipsum

\setlength{\headheight}{30.0pt}
\setlength{\footskip}{20pt}


\usepackage{hyperref}
\hypersetup{
    colorlinks=True,
    linkcolor={blue!20!black},
    filecolor=magenta,      
    urlcolor=cyan,
}



 \usepackage[export]{adjustbox}
\usepackage[english]{babel}
\usepackage[utf8]{inputenc}
\usepackage{fancyhdr}
\usepackage{multicol}

\pagestyle{fancy}
\fancyhf{}
\rhead{\textit{Pul074BEX004}}
\lhead{\textit{Amrit Prasad Phuyal}}
\rfoot{\thepage}


\usepackage{mathpazo} % Palatino font
\usepackage{graphicx}
\usepackage{float}


%%%%%% include  Titles.%%%% use \input{./CP}%%%
%%%use """"""""    \CP{}{}{}{}   """" %%%% and 4 argument to craete Title page 
%%%%%%%%%%%%%%%%%%%%%%%%%%%%%%%%%%%%%%%%%%%%%%%%%%%%%%%%%%%%%%%%%
%%%argument number
%% 1=major header ## Course name 
%% 2=minor4 heading ## lab/assignmet no
%% 3=Title  ## Assignment or Lab title
%% 4=submitted to::## input receiver Name"
%%%%%%%%%%%%%%%%%%%%%%%%%%%%%%%%%%%%%%%%%%%%%%%%%%%%%%%%%%%%%%%%%


\usepackage{mathpazo} % Palatino font
\usepackage{graphicx}
\usepackage{float}

%%% format and command for lab ans c and assembly

\newcommand{\HRule}{\rule{\linewidth}{0.4mm}} % Defines a new command for horizontal lines, change thickness here



%----------------------------------------------------------------------------------------
%	TITLE PAGE
%----------------------------------------------------------------------------------------


\newcommand{\CP}[4]{ \begin{titlepage} % Suppresses displaying the page number on the title page and the subsequent page counts as page 1
		%%%%  university logo%%
		\begin{figure}[H]
			\centering
			\includegraphics[scale=0.13]{tulogo.jpg}
		\end{figure}
		%%% end university logo

		\center % Centre everything on the page

		%------------------------------------------------
		%	Headings
		%------------------------------------------------

		\textsc{\huge Institute of Engineering \\ Central Campus,Pulchowk}\\[1.5cm] % Main heading such as the name of your university/college

		\textsc{\Large #1}\\[0.5cm] % Major heading such as course name

		\textsc{\large #2}\\[0.5cm] % Minor heading such as assignment no./ lab no.

		%------------------------------------------------
		%	Title
		%------------------------------------------------

		\HRule\\[0.4cm]

		{\Huge\bfseries #3}\\[0.4cm] % Title of your document

		\HRule\\[1.5cm]

		%------------------------------------------------
		%	Author(s)
		%------------------------------------------------
		\vfill\vfill
		\begin{minipage}{0.4\textwidth}
			\begin{flushleft}
				\large{
				\textbf{Submitted BY:}\\
				{\normalsize AMRIT PRASAD PHUYAL}\\ % NAME
				{\normalsize Roll: PULL074BEX004}} % Roll
			\end{flushleft}
		\end{minipage}
		~
		\begin{minipage}{0.4\textwidth}
			\begin{flushright}
				\large
				\textbf{Submitted To:}\\
				{ \normalsize{#4}\\ }% recepent's  Name 
				{\normalsize Department of Electronics and Computer Engineering}
			\end{flushright}
		\end{minipage}

		%------------------------------------------------
		%	Date
		%------------------------------------------------

		\vfill\vfill\vfill % Position the date 3/4 down the remaining page

		{\large\today} % Date, change the \today to a set date if you want to be precise

		\vfill % Push the date up 1/4 of the remaining page

	\end{titlepage}
} %%% cover page



%%%%%%%%% use  


%%%%%%%%% use  


%%%%%%%%% use  \input{Matlab.tex}
%%%%%%%%% use \MAT{###filename}{##Caption}
\usepackage{listings}

\usepackage{mdframed}
\usepackage{xcolor}
\definecolor{codegreen}{rgb}{0.1,0.9,0.2}
\definecolor{num}{rgb}{0.1,0.2,0.6}
\definecolor{keyword}{rgb}{0.1,0.1,0.99}
\definecolor{blackcolour}{rgb}{0,0,0}
\definecolor{string}{rgb}{0.9078,0.2,0.32}


\definecolor{front}{RGB}{8,39,20}
\definecolor{back}{RGB}{10,45,98}

% \renewcommand{\lstlistlistingname}{List of MATLAB codes}
% \renewcommand{\lstlistingname}{Code}

\renewcommand{\lstlistlistingname}{List of Matlab codes}
\renewcommand{\lstlistingname}{Code}


\lstdefinestyle{customa}{
    language=Matlab,
    % morekeywords={zeros,length},
    keywordstyle=\color{keyword},
    numberstyle=\tiny\color{num},
    stringstyle=\color{string},
    commentstyle=\color{codegreen},
    basicstyle=\ttfamily\footnotesize\color{front},
    breakatwhitespace=false,
    breaklines=true,
    captionpos=b,
    keepspaces=true,
    numbers=left,
    numbersep=15pt,
    showspaces=false,
    showstringspaces=false,
    showtabs=false,
    tabsize=4
}

\newcommand {\MAT}[2]{

    \begin{mdframed}[innerbottommargin=-2.3em,innertopmargin=-0.1em]
        \lstinputlisting[style=customa,caption={#2}]{#1}
    \end{mdframed}
}


% backgroundcolor=blueback,

%   innerbottommargin=-2.3em,innertopmargin=-0.1em,
%%     [outermargin =+1cm,]
%%%%%%%%% use \MAT{###filename}{##Caption}
\usepackage{listings}

\usepackage{mdframed}
\usepackage{xcolor}
\definecolor{codegreen}{rgb}{0.1,0.9,0.2}
\definecolor{num}{rgb}{0.1,0.2,0.6}
\definecolor{keyword}{rgb}{0.1,0.1,0.99}
\definecolor{blackcolour}{rgb}{0,0,0}
\definecolor{string}{rgb}{0.9078,0.2,0.32}


\definecolor{front}{RGB}{8,39,20}
\definecolor{back}{RGB}{10,45,98}

% \renewcommand{\lstlistlistingname}{List of MATLAB codes}
% \renewcommand{\lstlistingname}{Code}

\renewcommand{\lstlistlistingname}{List of Matlab codes}
\renewcommand{\lstlistingname}{Code}


\lstdefinestyle{customa}{
    language=Matlab,
    % morekeywords={zeros,length},
    keywordstyle=\color{keyword},
    numberstyle=\tiny\color{num},
    stringstyle=\color{string},
    commentstyle=\color{codegreen},
    basicstyle=\ttfamily\footnotesize\color{front},
    breakatwhitespace=false,
    breaklines=true,
    captionpos=b,
    keepspaces=true,
    numbers=left,
    numbersep=15pt,
    showspaces=false,
    showstringspaces=false,
    showtabs=false,
    tabsize=4
}

\newcommand {\MAT}[2]{

    \begin{mdframed}[innerbottommargin=-2.3em,innertopmargin=-0.1em]
        \lstinputlisting[style=customa,caption={#2}]{#1}
    \end{mdframed}
}


% backgroundcolor=blueback,

%   innerbottommargin=-2.3em,innertopmargin=-0.1em,
%%     [outermargin =+1cm,]
%%%%%%%%% use \MAT{###filename}{##Caption}
\usepackage{listings}

\usepackage{mdframed}
\usepackage{xcolor}
\definecolor{codegreen}{rgb}{0.1,0.9,0.2}
\definecolor{num}{rgb}{0.1,0.2,0.6}
\definecolor{keyword}{rgb}{0.1,0.1,0.99}
\definecolor{blackcolour}{rgb}{0,0,0}
\definecolor{string}{rgb}{0.9078,0.2,0.32}


\definecolor{front}{RGB}{8,39,20}
\definecolor{back}{RGB}{10,45,98}

% \renewcommand{\lstlistlistingname}{List of MATLAB codes}
% \renewcommand{\lstlistingname}{Code}

\renewcommand{\lstlistlistingname}{List of Matlab codes}
\renewcommand{\lstlistingname}{Code}


\lstdefinestyle{customa}{
    language=Matlab,
    % morekeywords={zeros,length},
    keywordstyle=\color{keyword},
    numberstyle=\tiny\color{num},
    stringstyle=\color{string},
    commentstyle=\color{codegreen},
    basicstyle=\ttfamily\footnotesize\color{front},
    breakatwhitespace=false,
    breaklines=true,
    captionpos=b,
    keepspaces=true,
    numbers=left,
    numbersep=15pt,
    showspaces=false,
    showstringspaces=false,
    showtabs=false,
    tabsize=4
}

\newcommand {\MAT}[2]{

    \begin{mdframed}[innerbottommargin=-2.3em,innertopmargin=-0.1em]
        \lstinputlisting[style=customa,caption={#2}]{#1}
    \end{mdframed}
}


% backgroundcolor=blueback,

%   innerbottommargin=-2.3em,innertopmargin=-0.1em,
%%     [outermargin =+1cm,] %%% Matlab code

\newcommand\ddfrac[2]{\frac{\displaystyle #1}{\displaystyle #2}} 



%%%%%%%%%%%%%%%%%%%%%for matlab observation #1 fig name #2 Caption
\newcommand{\mobs}[2]{
    \begin{figure}[H]
        \centering
        \includegraphics[width=1.07\linewidth]{./FIG/#1.eps}
        \caption{#2}
    \end{figure}
   
}




\begin{document}


%%%%  COver page 
\CP{Digital Signal Processing}{Lab \#2}{Familiarization with basic CT/DT functions}
{Anila  Kansakar}
%%%%%%%%%%%%%%%%%%%%

\pagenumbering{gobble}
\renewcommand{\contentsname}{Table of Contents}
\tableofcontents

\pagebreak
%\listoffigures
% \pagebreak
% \vspace{5em}
\lstlistoflistings
\vspace{10em}
% \pagebreak
\listoffigures
\pagebreak
\pagenumbering{arabic}

%%%%%%%%%%%%%%%%%%%%%%%%%%%%%%%%%%%%%%%%%%%%%%
\section{Title} {\large Familiarization with basic CT/DT functions }
%%%%%%%%%%%%%%%%%%%%%%%%%%%%
\section{Objective}
Familiarization with basic CT/DT functions

%%%%%%%%%%%%%%%%%%%%%


%Theory
\section{Theory}



\begin{verbatim}
    who         >> List variables in workspace
    whos        >> List variables in workspace, with sizes and types
    input()     >> Read input from user
    disp()      >> Display value of variable
    subplot()   >> Plot multiple graphs in one figure
    figure()    >> Create new figure window
    clear all   >> Clear all variables from workspace, freeing up system memory
    close all   >> Close all figures
    home        >> Send cursor to home position
    hold on     >> Retain plot data for multiple plots in one figure
    grid on     >> Turn on grid lines
    grid off    >> Turn off grid lines 
    grid        >> Turn on and off grid lines and set grid spacing
    demo        >> Access product examples in Help browser
    ver         >> Display version of Matlab and operating system information 
    lookfor     >> Search for keyword in all help files
    length()    >> Length of largest dimension of array
    pause       >> Stop MATLAB execution temporarily 
    plot()      >> 2-D line  plotting 
    stem()      >> Plot discrete data as stems
    real()      >> Return real part of complex number
    imag()      >> Return imaginary part of complex number
    zeros()     >> Create array of all zeros 
    ones()      >> Create array of  all ones 
    exp()       >> Return exponential of complex number
    for         >> Loop to repeat specified numbers of times
    end         >> End  For loop 
    if-else     >> If-else statement execute if statement is true
\end{verbatim}

\section {Lab Problems}
%%%%%%%%%%%%Problem 1

\subsection{Problem 1}
\subsection*{Plot the basic signal using Matlab}

%%%%%%%%%%%%% Problem 1aaaaaaaaaaaaaa
\subsubsection{Impulse response}
Codes:
\MAT{./CODES/p1a.m}{Matlab code for plotting Impulse function }
\mobs{impulse}{Plot for Impulse function }


%%%%%%%%%%%%% Problem 1bbbbbbbbbbbb
\subsubsection{Unit step response}
\MAT{./CODES/p1b.m}{Matlab code for plotting Unit Step function }
\mobs{unit}{Plot for Unit Step function }


%%%%%%%%%%%%% Problem 1ccccccccccccccc
\subsubsection{Ramp response}
\MAT{./CODES/p1c.m}{Matlab code for plotting Ramp function }
\mobs{ramp}{Plot for Ramp function}


%%%%%%%%%%%% Problem 1dddddddddddddd
\subsubsection{ Rectangular pulse response}
\MAT{./CODES/p1d.m}{Matlab code for plotting  Rectangular pulse function }
\mobs{rect}{Plot for Rectangular pulse function}

%%%%%%%%%%%%Problem 2
\subsection{Problem 2}
\subsection*{Plot the following continuous-time signals}


%%%%%%%%%%%%% Problem 2aaaa
\subsubsection*{$x(t)=Ce^{at}$ where $C$ and $a$ are real numbers and choose $C$ and $a$ both positive and negative.}
\MAT{./CODES/p2a.m}{Matlab code for C and a  Both real }
\mobs{ca_real}{Plot for for C and a  Both real}

%%%%%%%%%%%%% Problem 2bbbbbb
\subsubsection*{Plot the same signal taking $a$ as pure imaginary number}
\MAT{./CODES/p2b.m}{Matlab code for C real a Imaginary }
\mobs{a_imag}{Plot for C real a Imaginary}


%%%%%%%%%%%%% Problem 2ccccc
\subsubsection*{Consider complex exponential signal as specified in b) where $C$ is expressed in polar form i.e., $C=|C|e^{j\theta}$ and $a$ in rectangular form i.e., $a=r+j\omega_o$. Then function $x(t)$, on simplification, becomes $$ x(t)= |C|e^{rt}[\cos(\omega_o t+\theta)+j\sin(\omega_o t+\theta)]$$
Now, plot the signal for different values of r and comment on the results.\\
i. r=0 \quad \quad ii. r$<$ 0 \quad \quad iii. r$>$0}
\MAT{./CODES/p2c.m}{Matlab code for different value of r}


\subsubsection{For r=0}
\mobs{r0}{Plot for r=0}

\subsubsection{For r$<$0}
\mobs{rm1}{Plot for r$<$0}

\subsubsection{For r$>$0}
\mobs{r1}{Plot for r$>$0}


%%%%%%%%%%%%Problem 3
\subsection{Problem 3}
\subsection*{Plot the DT exponential function $x[n]=a^n$, $a=|a|e^{j\theta}$. Choose the suitable value of $|a|$ and $\theta$.}

\MAT{./CODES/p3.m}{Matlab code for calculation and plot DT exponential function}

\mobs{dt}{Plot for DT exponential function}


% %%%%%%%%%%%%Problem 4

\subsection{Problem 4}
\subsection*{Synthesize the signal from the FS coefficients as $C_0=1$, $C_1=C_{-1}=\ddfrac{1}{4}$, $C_2=C_{-2}=\ddfrac{1}{2}$, $C_3=C_{-3}=\ddfrac{1}{3}$.}
\MAT{./CODES/p4.m}{Matlab code for synthesizing and plotting signal}
\mobs{syn}{Plot for synthesized signal}


% %%%%%%%%%%%%Problem 5

\subsection{Problem 5}
\subsection*{Plot fundamental sinusoidal signal, its higher harmonics up to 5\textsuperscript{th} harmonics and add all of them to see the result. Comment on the result.}
\MAT{./CODES/p5.m}{Matlab code for calculation and plot of Sinusoidal harmonics and thier sum}
\mobs{harmonics}{Plot for Sinusoidal harmonics and thier sum}




%%Discussion and Conclusion
\section{Discussion and Conclusion}
In this Lab we familiarize ourself with Matlab Programming with basic of Continous time and Discrete time signals. We also learn about their basic operations in Matlab including  plotting and analyzing the signals. We learn to use online help for different commands and use them to calculate different DT and CT functions and plot them.


\end{document}