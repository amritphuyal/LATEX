%%%%%%%Circuit for Lab 3
\usepackage{tikz}
\usepackage{circuitikz}
\def\Res #1{
    \SI{#1}{\ohm}
}

\def\Resk #1{
    \SI{#1}{\kilo\ohm}
}

\def\Cap #1{
    \SI{#1}{\farad}
}

\def\Capu #1{
    \SI{#1}{\micro\farad}
}
\def\Capn #1{
    \SI{#1}{\nano\farad}
}
\def\Capp #1{
    \SI{#1}{\pico\farad}
}
\def\Ind #1{
    \SI{#1}{\henry}
}
\def\Indm #1{
    \SI{#1}{\milli\henry}
}

\def\Freq #1{
    \SI{#1}{\hertz}
}
\def\Freqk #1{
    \SI{#1}{\kilo\hertz}
}

\def\TheCkt #1#2#3#4#5#6#7{
    \begin{figure}[H]
        \centering
        \begin{circuitikz}[american]
            \draw
            (0,4) to [resistor, l^=${R_1=}${#3}] (4,4)
            to [inductor,l^=${L_1=}${#4}] (8,4)
            to [capacitor, l_=${C=}${#5},*-*] (8,0)
            to [short] (0,0)

            (8,4) to [inductor,l^=${L_2=}${#6}] (12,4)
            to [resistor,l_=${R_2=}${#7}] (12,0)
            to [short] (3,0)

            (0,4) to [esource, v_=$V_1$] (0,0)
            (13.5,4) to [open, v>=$V_2$] (13.5,0);
        \end{circuitikz}
        \caption{#1}
        \label{fig:#2}
    \end{figure}
}
